\begin{abstractgr}

Η ασφάλεια είναι ένα από τα βασικά χαρακτηριστικά κάθε σύγχρονου υπολογιστικού
συστήματος. Η παρούσα εργασία ερευνά επιθέσεις πάνω σε συμπιεσμένα
κρυπτογραφημένα πρωτόκολλα.

Συγκεκριμένα, προτείνεται μια νέα ιδιότητα που χαρακτηρίζει κρυπτοσυστήματα, η
μη-διακρισιμότητα ενάντια σε επιθέσεις μερικώς επιλεγμένου κειμένου (IND-PCPA),
καθώς και ένα μοντέλο επίθεσης που χρησιμοποιεί αυτή την ιδιότητα. Προκειμένου
να ξεπεραστούν εμπόδια που παρουσιάζονται σε συστήματα του πραγματικού κόσμου,
προτείνονται στατιστικές μέθοδοι, οι οποίες βελτιώνουν την επίδοση και
εγκυρότητα της επίθεσης.

Τα πειράματα που διεξήχθησαν κατά τη διάρκεια της εργασίας αφορούσαν σε δύο
ευρέως χρησιμοποιούμενα συστήματα, το Facebook Chat και το Gmail, για την
επίτευξη των οποίων χρησιμοποιήθηκε λογισμικό το οποίο αναπτύχθηκε σε Python
για τους σκοπούς αυτής της εργασίας. Τα πειράματα έγιναν σε συνθήκες εργαστηρίου
και απέδειξαν ότι τα δύο αυτά συστήματα δεν είναι IND-PCPA, όσον αφορά
συγκεκριμένους τύπους μυστικών.

Τέλος, προτείνονται καινοτόμες τεχνικές, οι οποίες θα οδηγήσουν σε πλήρη
αντιμετώπιση επιθέσεων που ακολουθούν το μοντέλο που προτείνεται, όπως η
επίθεση που παρουσιάστηκε στην παρούσα εργασία.

\end{abstractgr}

\begin{abstracten}

Security is a fundamental aspect of every modern system. This work investigates
attacks on compressed encrypted protocols.

A new property of cryptosystems is proposed, called Indistinguishability under
Partially Chosen Plaintext Attack (IND-PCPA), along with an attack model that
works under such a mechanism. In order to bypass obstacles of real-world
systems, statistical methods were proposed, to improve the performance and
validity of the attack.

Experiments were conducted on two widely used systems, Facebook Chat and Gmail, using
a Python framework that was implemented for the purpose of this paper. Results
in lab environment revealed that those two systems are not IND-PCPA, regarding
certain types of secrets.

Finally, novel techniques were proposed, that could lead to complete mitigation
of attacks that follow the proposed model.

\end{abstracten}

\begin{acknowledgementsgr}
Η παρούσα διπλωματική εργασία εκπονήθηκε στα πλαίσια της φοίτησής μου στο τμήμα
Ηλεκτρολόγων Μηχανικών και Μηχανικών Υπολογιστών του Εθνικού Μετσόβιου
Πολυτεχνείου.

Η διπλωματική αυτή εκπονήθηκε υπό την επίβλεψη του καθηγητή Αριστείδη Παγουρτζή,
τον οποίο θα ήθελα να ευχαριστήσω θερμά για τη βοήθειά του, καθώς και για το
γεγονός ότι μέσω της διδασκαλίας της Κρυπτογραφίας με εισήγαγε στο αντικείμενο
και με οδήγησε στον τομέα της ασφάλειας.

Ακόμα, θα ήθελα να ευχαριστήσω τον Διονύση Ζήνδρο, ο οποίος αρχικά μου πρότεινε
το θέμα της εργασίας και στη συνέχεια με κατεύθυνε, με συμβούλευε και αφιέρωσε
πολύ χρόνο για να συζητήσουμε τα βασικά σημεία της.

Επιπλεόν, θα ήταν παράλειψη να μην ευχαριστήσω τον Angelo Prado, εκ των
δημιουργών της αρχικής επίθεσης BREACH, για την αμέριστη βοήθειά του στα
προβλήματα που αντιμετωπίσαμε και στη συνολική υλοποίηση της επίθεσης.

Τέλος, θα ήθελα να ευχαριστήσω τους φίλους και την οικογένειά μου για τη στήριξη
που μου παρείχαν όλα αυτά τα χρόνια.

\end{acknowledgementsgr}
