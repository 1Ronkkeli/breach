\chapter{Conclusion}\label{conclusion}

\section{Concluding remarks}

Attacks on encrypted protocols that exploit compression methods applied on the
plaintext handled by those protocols have only recently been described.
Literature so far shows limited theoretical definitions of this new type of
attacks, while experimental results relate to a relatively small scope of
protocols used nowadays.

This work focused on assessing the threat of such attacks for widely used
protocols by expanding the theoretical definition and investigated the
success of methods designed for mitigation.

We introduced a cryptographical game for determining the property of
indistinguishability under partially chosen plaintext attacks. Also we provided
intuitive proofs for comparison to other indistinguishability properties, along
with scenarios of application of partially chosen plaintext attacks on
compressed encrypted protocols.

The need for practical description of our method resulted in the definition of
an attack model based on BREACH that initiates, automates and validates the
attack. We also revealed major vulnerabilities on the two systems that we
experimented on, Facebook Chat and Gmail, introducing new forms of secrets and
chosen plaintext an attacker could use.

We expanded the scope of the attack to block ciphers and we ulitized various
statistical methods that bypass known obstacles such as noise and padding.
Furthermore we proposed various optimization techniques that could reduce the
time and increase the efficiency of the attack posing a valid threat for
real-world systems.

In order to perform experiments and validate the efficiency of the attack, we
implemented a framework in Python that initiates the attack on a chosen
endpoint and parses the output in order to produce statistical results. From an
attacker's perspective, the framework must run on a machine inside the victim's
network, while the victim's machine is configured to send all traffic to the
endpoint to the attacker's machine and the victim also browses a website
controlled by the attacker.

Experimental results have shown that although the framework does not provide a
robust functionality, the attacker has a considerable advantage on
stealing a secret from the endpoints tested.

Finally we investigated the ability of previously proposed mitigation techniques
to stop the attack and proposed novel methods that could effectively minimize
the attack's success or even mitigate it completely.

\section{Future Work}

Although this work introduced the IND-PCPA property, formal definitions and
mathematical description is still necessary. Also this new property should be
formally evaluated compared to other known properties.

As far as the practical part of the attack, a consistency mechanism as
described in Section \ref{sec:persistence} is needed in order to take full
advantage of vulnerabilities of simple HTTP connections. Furthermore the
integration of MitM attacks, like the ones referenced in Section \ref{sec:mitm},
would result in a potential threat outside lab environment. It is also important
to implement a MitM proxy on TCP level that would be able to distinguish
packets of different records, minimizing the margin of error for overlapping
response or request streams.

Finally implementation of the two proposed mitigation techniques,
compressibility annotation [\ref{subsec:annotation}] and SOS headers
[\ref{subsec:sos}], is vital in order to secure systems against attacks that
utilize the findings of this work.
