\chapter{Partially Chosen Plaintext Attack}\label{ch:pcpa}

Traditionally, cryptographers have used games for security analysis. Such games
include the indistinguishability under chosen plaintext attack (IND-CPA), the
indistinguishability under chosen ciphertext attack/adaptive chosen ciphertext
attack (IND-CCA1, IND-CCA2)
etc\footnote{\url{https://en.wikipedia.org/wiki/Ciphertext_indistinguishability}}.

In this chapter, we introduce a definition for a new property of encryption
schemes, called indistinguishability under partially-chosen-plaintext-attack
(IND-PCPA). We also provide comparison between IND-PCPA and other known forms of
cryptosystem properties.

\section{Partially Chosen Plaintext Indistinguishability}\label{sec:indpcpa}

\subsection{Definition}

IND-PCPA uses a definition similar to that of IND-CPA.

For a probabilistic asymmetric key encryption algorithm, indistinguishability
under partially chosen plaintext attack (IND-PCPA) is defined by the following
game between an adversary and a challenger.

\begin{itemize} \item The challenger generates a key pair \begin{math}P_k,
S_k\end{math} and publishes \begin{math}P_k\end{math} to the adversary.  \item
The adversary may then perform a polynomially bounded number of encryptions or
other operations.  \item Eventually, the adversary submits two distinct chosen
plaintexts \begin{math}M_0, M_1\end{math} to the challenger.  \item The
challenger selects a bit \begin{math}b\in{0, 1}\end{math} uniformly at random.
\item The adversary can then submit any number of selected plaintexts
\begin{math}R_i, i\in N, |R| \geq 0\end{math}, for which the challenger sends
the ciphertext \begin{math}C_i = E(P_k, M_b||R_i)\end{math} back to the
adversary.  \item The adversary is free to perform any number of additional
computations or encryptions, before finally guessing the value of
\begin{math}b\end{math}.  \end{itemize}

A cryptosystem is indistinguishable under partially chosen plaintext attack, if
every probabilistic polynomial time adversary has only a negligible advantage on
finding \begin{math}b\end{math} over random guessing. An adversary is said to
have a negligible advantage if it wins the above game with probability
\begin{math}\frac{1}{2} + \epsilon(k)\end{math}, where
\begin{math}\epsilon(k)\end{math} is a negligible function in a security
parameter \begin{math}k\end{math}.

Intuitively, we can think of the adversary as having the ability to modify the
plaintext of a message, by appending a chosen portion of data to it, without
prior knowledge of the plaintext itself. He can then acquire the ciphertext of
the modified text and perform any kinds of computations on it. A system would
then be described as IND-PCPA, if the adversary is unable to gain more
information about the plaintext, than he would by guessing at random.

\subsection{IND-PCPA vs IND-CPA}

Suppose the adversary submits the empty string as the chosen plaintext, a choice
which is allowed by the definition of the game. The ciphertext that the
challenger would then send back would be \begin{math}C_i = E(P_k, M_b||"\ ") =
E(P_k, M_b)\end{math}, which is the ciphertext returned from the challenger in
the context of the IND-CPA game.

Therefore, if the adversary has the ability to beat the game of IND-PCPA, i.e.
if the system is not indistinguishable under partially chosen plaintext attacks,
he also has the ability to beat the game of IND-CPA. This assumption
provides an informal proof that IND-PCPA is at least as strong as IND-CPA.

\section{PCPA on compressed encrypted protocols}\label{sec:cepcpa}

In this section we will investigate the relationship between compression and
encryption, regarding how partially chosen plaintext attacks can exploit either
one method in protocols that allow such functionality schemes.

\subsection{Compression-before-encryption and vice versa}

When having a system that applies both compression and encryption on a given
plaintext, it would be interesting to investigate the order those
transformations should be executed.

Lossless compression algorithms rely on statistical patterns to reduce the size
of the data to be compressed, without losing information. Such a method is
possible, since most real-world data present statistical redundancy. However, it
can be understood that such compression algorithms will fail to compress well
certain kinds data sets, that display no statistical pattern.

Encryption algorithms, on the other hand, rely on adding entropy to the
plaintext before producing the ciphertext. If the ciphertext contained repeated
portions, these statistical patterns could be exploited in order to deduce
information about the plaintext.

In the scheme where we apply compression after encryption, the ciphertext to be
compressed should demonstrate no statistical analysis exploits, not allowing
compression to reduce the size of the data. In addition, compression after
encryption would not increase the security of the protocol.

On the other contrary, applying encryption after compression seems a more
preferable solution. The compression algorithm can use the statistical
redundancies of the plaintext to perform well, while the encryption algorithm,
if applied correctly on the compressed text, should produce a random stream of
data. Also, since compression introduces additional entropy, this scheme
should make it harder for attackers, who rely on differential cryptanalysis,
to break the system.

\subsection{PCPA scenario on compression-before-encryption protocol}

Let's assume a system that composes encryption and compression in the following
manner:

\begin{math}c = Encrypt(Compress(m))\end{math}

where \begin{math}c\end{math} is the ciphertext and \begin{math}m\end{math} is
the plaintext.

Suppose the plaintext contains a specific secret, among random strings of data,
and the attacker can issue a PCPA with a chosen plaintext, which we will call
reflection. The plaintext then takes the form:

\begin{math}m = n_1 || secret || n_2 || reflection || n_3\end{math}

where \begin{math}n_1, n_2, n_3\end{math} are random nonces.

If the reflection is the same as the secret, the compression mechanism will
recognize this pattern and compress the two data portions. In other case, the
two strings will not demonstrate any statistical redundancy and compression will
perform worse. As a result, in the first case the data to be encrypted will be
smaller than in the second case.

Usually encryption is done by a stream or a block cipher. In the first case, the
lengths of a plaintext and the corresponding ciphertext are identical, whereas
in the second case they differ by the number of padding bits, which is
relatively small. That way, for a system as the one mentioned, an adversary
could identify a pattern and extract information about the plaintext, based on
the lengths of the two ciphertexts.

\section{Known PCPA exploits}\label{sec:known_pcpa}

In this section, we cite known attacks that use the partially chosen plaintext
attack vector, in a context as described in the previous section.

\subsection{CRIME}

\texttt{Compression Ratio Info-leak Made Easy} (CRIME) \cite{crime} is a
security exploit that was revealed at the 2012
ekoparty\footnote{\url{https://www.ekoparty.org}}.  As stated, "it decrypts
HTTPS traffic to steal cookies and hijack sessions".

In order for the attack to succeed, there are two requirements. Firstly, the
attacker should be able to sniff the victim's network traffic, so as to see the
request/response packet lengths. Secondly, the victim should visit a website
controlled by the attacker or surf on non-HTTPS sites, in order for the CRIME
script to be executed.

If the above requirements are met, the attacker makes a guess for the secret to
be stolen and asks the browser to send a request with this guess included in the
path.  The attacker can then observe the length of the request and, if the
length is less than usual, it is assumed that the guessed string was compressed
with the secret, so it was correct.

CRIME has been mitigated by disabling TLS and SPDY compression on both Google
Chrome and Mozilla Firefox browsers, as well as various server software
packages. However, HTTP compression is still supported, while some web servers
that support TLS compression are also vulnerable.

\subsection{BREACH}

\texttt{Browser Reconnaissance and Exfiltration via Adaptive Compression of
Hypertext (BREACH)} \cite{breach} is a security exploit that is based on CRIME.
Presented at the August 2013 Black Hat
conference\footnote{\url{https://www.blackhat.com}}, it targets the size of
compressed HTTP responses and extracts secrets hidden in the response body.

Like the CRIME attack, the attacker needs to sniff the victim's network traffic,
as well as force the victim's browser to issue requests to the chosen endpoint.
Additionally, it works against stream ciphers only and assumes zero noise in the
response. Moreover, it requests a known prefix for the secret, although a
solution for this condition would be to guess the first two characters of the
secret, in order to bootstrap the attack.

From then on, the methodology is in general the same as CRIME's. The attacker
guesses a value, which is then included in the response body along with the
secret and, if correct, it is compressed well with it, resulting in smaller
response length.

BREACH has not yet been fully mitigated, although Gluck, Harris and Prado
proposed various counter measures for the attack. We will investigate these
mitigation techniques in depth in Chapter \ref{ch:mitigation}.
